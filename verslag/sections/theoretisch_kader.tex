\documentclass[../verslag.tex]{subfiles}
\graphicspath{{\subfix{../images/}}}
\begin{document}
Om een zinvol, accuraat model van een sluis te maken zijn er bepaalde aspecten van dit onderwerp die eerst uitgelicht moeten worden. Voor het bereiken van dit doel zijn een aantal deelvragen opgesteld.

Zo gaan wij onderzoeken wat een sluis is. Dit onderzoek focust zich op sluizen van de scheepvaart. Niet alleen wordt de definitie van de sluis verder uitgewerkt, er wordt ook gekeken naar sluizen van verschillende capaciteiten en functies. Het beantwoorden van deze deelvraag zal ons helpen om beslissingen te kunnen maken omtrent het model met betrekking tot welke onderdelen er zijn, en welke relevant zijn te modelleren.

Verder onderzoeken wij ook hoe een sluis opereert. Gezien dit onderzoek moet leiden tot een model die de (automatische) functie van een sluis demonstreert, is het belangrijk om te weten welke operaties uitgevoerd moeten worden om een sluis correct te laten functioneren. Het onderzoek van deze functionaliteiten zal later ook helpen met het bepalen van welke selectie hiervan relevant is om te includeren in het model.

Ook zal er onderzoek worden gedaan naar hoe sluizen onderhouden worden. Om eisen op te kunnen stellen met betrekking tot de duurzaamheid en onderhoud van een sluis is het relevant om te weten wanneer, waarom en waaraan een sluis onderhouden moet worden. De conclusie van deze deelvraag zal ook gebruikt worden om een licht te schijnen op hoe de twee eerder genoemde abstracte aspecten van het eisenpakket mogelijk gemodelleerd zouden kunnen worden.

Als laatste zullen wij ook onderzoek doen naar wat er allemaal fout kan gaan met een sluis. Wij hopen hiervan te leren wat de foutgevoelige onderdelen zijn van de sluis, en wat eraan gedaan kan worden om deze fouten te voorkomen. Deze informatie wordt hierna gebruikt om eisen op te stellen omtrent de veiligheid van het systeem, en er wordt mee rekening gehouden gedurende de ontwikkeling van het model.

Er is al eerder onderzoek gedaan naar het simuleren van sluizen. Zo is er in 1976 een onderzoek uitgevoerd naar het optimaliseren van het verkeer binnen een gesimuleerd kanalen netwerk \cite{oosterveld_1976}. Het onderzoek beschrijft het simulatie model, haar onderdelen, en andere specificaties als hoe schepen worden gegenereerd en het proces dat de sluis doorloopt. Dit model was gebouwd binnen de simulatietaal PROSIM. Het artikel beschrijft ook aan wat voor aspecten het model geoptimaliseerd kon worden om kosten, reistijd en benutting van kanalen te verbeteren. Dit laatste is minder relevant voor ons onderzoek. Het nut van dit onderzoek binnen dat van ons zou beperkt kunnen zijn door het feit dat het 46 jaar oud is.

\subsection{Begrippenlijst}
\emph{Verval}: "Verschil in hoogte van de waterspiegel op twee punten van een rivier" \cite{dvd_verval}\\
\emph{Sluis}: Een sluis is een scheiding tussen 2 wateren, met deuren. Hierdoor is het mogelijk het waterpeil te beïnvloeden. Sluizen reguleren het waterpeil zodat schepen kunnen passeren. \cite{rws_2022}
\end{document}
