\documentclass[../verslag.tex]{subfiles}
\graphicspath{{\subfix{../images/}}}
\begin{document}

In conclusie is er begonnen met een literatuuronderzoek naar de nederige schutsluis. Er is gekeken naar verschillende aspecten van dit infrastructurele wonder. Uit dit onderzoek zijn hierna eisen opgesteld die gebruikt zijn om een model van een sluis op te bouwen, en daarna richtlijnen te hebben om dit model op te kunnen beoordelen. Deze requirements zijn opgedeeld op basis van veiligheid, efficiëntie, capaciteit, onderhoud, onderhoudskosten en duurzaamheid. 

Hierna hebben is er een modulair model neer gezet van een schutsluis met deze voorgenoemde eisen in het achterhoofd, die functioneert op een tijdschema. Deze schutsluis beschikt alleen over hetgeen dat ook daadwerkelijk nodig is om een sluis van enige grootte te bedienen. De sluis beschikt over twee deuren, vanzelfsprekend een schutkolk, en signalering in de vorm van stoplichten. Deze onderdelen werken in harmonie samen door middel van het gebruik van een controller, de \verb|SLU15-W8r|.

We hebben voor dit model hierna verificaties geschreven in de vorm van computation tree logic statements. Hiermee hebben wij van ons model succesvol alle eisen kunnen verifiëren die niet gaan over dingen die voor ons buiten de scope vielen als onderhoud, onderhoudskosten en duurzaamheid. De capaciteit van het model is in ons geval algoritmisch bepaald, en het was voor deze eis niet nodig om additionele verificatie te gebruiken. Uit onze succesvolle verificaties kunnen wij concluderen dat onze sluis voldoet aan de relevante requirements. Het verifiëren van de niet meegenomen eisen zal op een later stadium moeten worden gedaan, in een ander onderzoek. Dit onderzoek kan wel bieden als een richtlijn voor het verder verifiëren van deze eisen.

\end{document}
