\documentclass[../verslag.tex]{subfiles}
\graphicspath{{\subfix{../images/}}}
\begin{document}
In dit onderzoek is er kwalitatief en kwantitatief onderzoek uitgevoerd om antwoord te geven op de vraag hoe er een model gemaakt kan worden voor een watersluis systeem die voldoet aan de vereisten van de ambtenaar.
Hiervoor is literatuuronderzoek gedaan en zijn vervolgens modellen gemaakt aan de hand van dit literatuuronderzoek.

\subsection{Dataverzameling}
Voor het literatuuronderzoek zijn er bronnen en artikelen gezocht via Google die betrekking hebben tot watersluizen en de vereisten van de ambtenaar.

\subsection{Inclusie- en exclusiecriteria}
Voor het literatuuronderzoek zijn er gebruikgemaakt van bronnen en artikelen van de afgelopen 20 jaar. 
Bovendien is er ook gekeken naar watersluizen buiten Europa.
\subsection{Onderzoeksverloop}
Na vraag van de ambtenaar naar een model van een geautomatiseerde watersluis. TODO
\subsection{Data-analyse}
De verkegen informatie over de watersluizen uit veschillende bronnen, is met elkaar vergeleken.
\subsection{Validiteit en betrouwbaarheid}
Ter behoeve van de validiteit zijn de gemaakt modellen gebaseerd op de verkregen informatie uit het literatuuronderzoek.
Om de betrouwbaarheid van de modellen te garanderen, zijn de modellen geverifiërd aan de hand van de requirements die vertaald zijn naar ctl voor verificatie in Uppaal.
\end{document}
