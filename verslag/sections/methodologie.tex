\documentclass[../verslag.tex]{subfiles}
\graphicspath{{\subfix{../images/}}}
\begin{document}
In dit onderzoek is er kwalitatief en kwantitatief onderzoek uitgevoerd om antwoord te geven op de vraag hoe er een model gemaakt kan worden voor een sluis systeem die voldoet aan de vereisten van de ambtenaar.
Hiervoor is literatuuronderzoek gedaan en zijn vervolgens modellen gemaakt aan de hand van dit literatuuronderzoek.

\subsection{Dataverzameling}
Voor het literatuuronderzoek zijn er bronnen en artikelen gezocht via Google die betrekking hebben tot sluizen en de vereisten van de ambtenaar.
Daarbuiten is er ook gebruik gemaakt van de verschillende databases waarin de Hogeschool Rotterdam in voorziet.

\subsection{Inclusie- en exclusiecriteria}
Voor het literatuuronderzoek zijn er gebruikgemaakt van bronnen en artikelen die te vinden zijn op het web in het Nederlands of Engels.
Er is zich niet beperkt tot een bepaalde tijdsperiode aangezien sluizen een oud concept zijn en informatie hierover nog steeds relevant is ongeacht de tijd van schrijven.
Bovendien is er gekeken naar sluizen over de hele wereld en niet alleen in Europa.
\subsection{Onderzoeksverloop}
Na vraag van de ambtenaar naar een model van een geautomatiseerde sluis is er eerst een literatuuronderzoek verricht naar de verschillende aspecten van een sluis.
Dit literatuuronderzoek is onderbouwd door verschillende relevante bronnen die te vinden zijn over het onderwerp.
Het vinden van verschillende soorten situaties bij de mogelijke sluisrampen was lastig aangezien de meeste rampen in dezelfde categorie vallen.
Hierdoor moest het aantal situaties beperkt worden.
\subsection{Data-analyse}
De verkegen informatie over de sluizen uit veschillende bronnen, is met elkaar vergeleken.
\subsection{Validiteit en betrouwbaarheid}
Ter behoeve van de validiteit zijn de gemaakte modellen gebaseerd op de verkregen informatie uit het literatuuronderzoek.
Om de betrouwbaarheid van de modellen te garanderen, zijn de modellen geverifiërd aan de hand van de requirements die vertaald zijn naar ctl voor verificatie in Uppaal.
Aangezien niet alle requirements naar ctl vertaald kunnen worden, is het verificatieproces beperkt tot de requirements die wel vertaald kunnen worden naar ctl.
Vervolgens is er gereflecteerd of het verificatieproces juist is verlopen en of er mogelijke verificaties niet gelukt zijn.

\end{document}
