\documentclass[../verslag.tex]{subfiles}
\graphicspath{{\subfix{../images/}}}
\begin{document}

\subsection{Wat is een sluis?}

\subsection{Hoe opereert een sluis?}

\subsection{Hoe wordt een sluis onderhouden?}

\subsection{Wat kan er fout gaan met een sluis?}
Bij het modelleren van een systeem is het handig om rekening te houden met dingen die mis kunnen gaan in het systeem.
Dit kan betrekking hebben tot hardware, software, omstandigheden van buitenaf maar ook een menselijke fout.
Enkele van deze gevallen zullen daarom besproken worden zodat ,in dien dit mogelijk is, deze kennis te gebruiken bij het modelleren.

Op 27 september 2009 was er een storing bij de Markland Locks and Dam op de Ohio rivier. 
Hierbij werkte één van de poorten niet zoals behoren vanwege gebreken in de onderdelen van de poort\cite{herald-dispatch_2009}.  
De poort ging hierdoor niet dicht waardoor water direct weer weg kon stromen.
Deze gebreken kwamen door de beperkte onderhoud die werd gedaan aan deze watersluis.

Vanwege de droogte in de zomer en herfstperiode in het jaar 2018 waren er enkele potentieele gevaren gemeld bij watersluizen in Nederland\cite{poelgeest_2020}.
Door de lage waterpeil kon er niet gegarandeerd worden dat de sluizen volledig intact kon blijven.
Daarom was de keuze gemaakt om een aantal van deze sluizen buiten bedrijf te stellen.
Anderen sluizen werden wel operationeel gehouden ondanks de mogeljke gevaren dat dit met zich mee bracht.

Problemen in een watersluis systeem vallen over het algemeen in deze twee categorieën. 
Deze twee categorieën zijn: storingen vanwege onderdelen die niet onderhouden zijn en externe factoren zoals een buitengewone waterpeil.
Het modelleren van deze categorieën kan een uitdaging zijn omdat deze problemen niet direct te maken hebben met het systeem zelf.
Er kan eventueel rekening gehouden worden met factoren zoals waterpeil bij het modelleren van een watersluis. 
Zoals het automatisch buitenwerking stellen van een watersluis bij een te lage waterpeil. 
Gebreken aan fysieke onderdelen kunnen moeilijk gemodelleerd worden en hoeven niet relevant te zijn voor het model.



\end{document}
