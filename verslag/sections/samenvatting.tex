\documentclass[../verslag.tex]{subfiles}
\graphicspath{{\subfix{../images/}}}
\begin{document}

Dit verslag begint met een literatuuronderzoek over wat een sluis is, hoe een sluis opereert, het onderhoud van een sluis en wat er fout kan gaan met een sluis. Vervolgens zijn er uit dit onderzoek eisen opgesteld die gebruikt worden om het systeem te modelleren, en daarna richtlijnen te hebben om het model op te kunnen beoordelen. Deze requirements zijn opgedeeld op basis van veiligheid, efficiëntie, capaciteit, onderhoud, onderhoudskosten en duurzaamheid. 
Het modulaire model is neergezet met de voorgenoemde eisen en functioneert op een tijdschema. De sluis beschikt over twee deuren, een schutkolk en signalering in de vorm van stoplichten. Deze onderdelen zorgen voor de volledig werking van de sluis.
Voor het model zijn er verificaties geschreven in de vorm van computation tree logic statements. Hiermee hebben wij van ons model succesvol alle eisen kunnen verifiëren die niet gaan over dingen die voor ons buiten de scope vielen als onderhoud, onderhoudskosten en duurzaamheid. De capaciteit voor het model is algoritmisch bepaald, en het was niet nodig om additionele verificatie te gebruiken.
Er zouden in de latere stadia van het bouwen van de sluis andere eisen geverifieerd moeten worden, die niet in dit onderzoek zijn meegenomen. Dit onderzoek kan wel bieden als een richtlijn voor het verder verifiëren van deze eisen.

\end{document}
