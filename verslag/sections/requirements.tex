\documentclass[../verslag.tex]{subfiles}
\graphicspath{{\subfix{../images/}}}
\begin{document}
Op basis van ons onderzoek hebben we requirements opgesteld om te kunnen voldoen aan de eisen van het ministerie. Deze zijn gebaseerd op het literatuur onderzoek. De eisen zijn hieronder gecategoriseerd te vinden.

\subsection{Veiligheid}
De veiligheid van de schippers en betrokkenen staat boven alles. De sluis moet zig aan een aantal regels houden om rampen te voorkomen. Ook moet het gebruik van de sluis duidelijk zijn voor de schippers.
\begin{itemize}
    \item Sluisdeuren zijn dicht bij op- en neerschutten.
    \item Stoplichten staan altijd op rood/groen bij relevante staten van de deur.
    \item Deuren staan nooit tegelijkertijd open.
    \item De sluis is voorzien van waarschuwingsborden.
\end{itemize}

\subsection{Efficiëntie}
De boten moeten efficiënt gebruik kunnen maken van een sluis. Door een tijdschema te gebruiken kunnen de schippers hier rekening mee houden en zo hun tijd goed inplannen. Zo kunnen de boten ter plaatse zijn zonder lang voor de sluis te hoeven wachten.
\begin{itemize}
    \item De sluis cycled binnen x minuten.
    \item De sluis cycled op basis van CEMT-klasse.
    \item De sluis cycled niet meer dan nodig.
    \item De sluis werkt op basis van een tijdschema.
    \item Iedereen mag eerst de sluis uit voordat de sluis weer vol loopt.
\end{itemize}

\subsection{Capaciteit}
De capaciteit is op basis van CEMT-klasse en op het referentieschip van de waterwegbeheerder. Zo is het aantal mogelijke boten vastgesteld met een algemeen bekende maatstaaf. De hoeveelheid boten kan in het model aangepast worden om te voldoen aan de wensen van het Ministerie van Infrastructuur en Waterstaat.
\begin{itemize}
    \item De sluis heeft de correcte capaciteit gebaseerd op CEMT-klasse.
    \item De sluis heeft de correcte capaciteit gebaseerd op het referentieschip van de waterwegbeheerder.
\end{itemize}

\subsection{Onderhoud}
Inspectie en onderhoud is belangrijk om de sluis operationeel te houden. Om dit zo makkelijk mogelijk te maken moeten de onderdelen makkelijk bereikbaar zijn. Ook moet de staat van de apparatuur digitaal gemeten worden en waarschuwingen geven aan de sluisbediende bij problemen.
\begin{itemize}
    \item De onderdelen van de sluis moeten bereikbaar zijn om dit makkelijk te kunnen onderhouden.
    \item De conditie van de apparatuur van de sluis moet automatisch gemeten worden met sensoren.
    \item De sluis bediende moet waarschuwingsmeldingen krijgen van afwijkingen in de apparatuur.
    \item Bij een defect onderdeel moet de sluis niet gebruikt kunnen worden.
\end{itemize}

\subsection{Onderhoudskosten}
Het is wenselijk de onderhoudskosten zo laag mogelijk te houden.
\begin{itemize}
    \item De onderhoudskosten moeten zo laag mogelijk zijn.
\end{itemize}

\subsection{Duurzaamheid}
Voor mens en natuur is het belangrijk de schade aan het milieu zo klein mogelijk te houden. De sluizen moeten duurzamer zijn voor het onlangs afgesloten klimaatakkoord.
\begin{itemize}
    \item De sluis moet lang mee kunnen gaan.
    \item De sluis vervuilt zo min mogelijk.
    \item De sluis breng zo min mogelijk schade aan de omringende natuur.
\end{itemize}

\end{document}
