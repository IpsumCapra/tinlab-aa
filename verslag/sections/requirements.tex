\documentclass[../verslag.tex]{subfiles}
\graphicspath{{\subfix{../images/}}}
\begin{document}
Op basis van ons onderzoek hebben we requirements opgesteld om te kunnen voldoen aan de eisen van het ministerie. Deze zijn gebaseerd op het literatuur onderzoek. De eisen zijn hieronder gecategoriseerd te vinden.

\subsection{Veiligheid}
\begin{itemize}
    \item Sluisdeuren zijn dicht bij op- en neerschutten.
    \item Stoplichten staan altijd op rood/groen bij relevante staten van de deur.
    \item Deuren staan nooit tegelijkertijd open.
    \item De sluis is voorzien van waarschuwingsborden.
\end{itemize}

\subsection{Efficiëntie}
\begin{itemize}
    \item De sluis cycled binnen x minuten.
    \item De sluis cycled op basis van CEMT-klasse.
    \item De sluis cycled niet meer dan nodig.
    \item De sluis werkt op basis van een tijdschema.
    \item Iedereen mag eerst de sluis uit voordat de sluis weer vol loopt.
\end{itemize}

\subsection{Capaciteit}
\begin{itemize}
    \item De sluis heeft de correcte capaciteit gebaseerd op CEMT-klasse.
    \item De sluis heeft de correcte capaciteit gebaseerd op het referentieschip van de waterwegbeheerder.
\end{itemize}

\subsection{Onderhoud}
\begin{itemize}
    \item De onderdelen van de sluis moeten bereikbaar zijn om dit makkelijk te kunnen onderhouden.
    \item De conditie van de apparatuur van de sluis moet automatisch gemeten worden met sensoren.
    \item De sluis bediende moet waarschuwingsmeldingen krijgen van afwijkingen in de apparatuur.
    \item Bij een defect onderdeel moet de sluis niet gebruikt kunnen worden.
\end{itemize}

\subsection{Onderhoudskosten}
\begin{itemize}
    \item De onderhoudskosten moeten zo laag mogelijk zijn.
\end{itemize}

\subsection{Duurzaamheid}
\begin{itemize}
    \item De sluis moet lang mee kunnen gaan.
    \item De sluis vervuilt zo min mogelijk.
    \item De sluis breng zo min mogelijk schade aan de omringende natuur.
\end{itemize}

\end{document}
