\documentclass[../verslag.tex]{subfiles}
\graphicspath{{\subfix{../images/}}}
\begin{document}

Rijkswaterstaat definieert een sluis als een scheiding tussen 2 wateren, met deuren, waarmee het mogelijk is om het waterpeil te beïnvloeden. Sluizen reguleren het waterpeil zodat schepen kunnen passeren. \cite{rws_2022}

In Nederland zijn veel verschillende soorten sluizen te vinden \cite{wsnoorderzijlvest_2021},\cite{arends_1994}. Zo zijn er doksluizen die worden gebruikt om het waterpeil in droogdokken en havens te regelen, en bijvoorbeeld spuisluizen die voorkomen dat de vaargeul waaraan ze zijn aangesloten dichtslibben door middel van een kom met zeewater die wordt gevuld tijdens vloed, en weer leeg wordt laten gelopen als het eb is \cite{wsnoorderzijlvest_2021},\cite{arends_1994}.

Dit onderzoek daarentegen focust zich op schutsluizen. Schutsluizen verschillen in veel andere sluizen in dat ze beschikken over twee of meer sluishoofden. Dit leidt ten gevolge ook tot dat schutsluizen een schutkolk hebben. De afgesloten ruimte die wordt gecreëerd als de keringen in de sluishoofden worden gesloten. De schutkolk kan gebruikt worden om geleidelijk het verval tussen de twee waterhoogten aan de buitenzijden van de sluishoofden te overbruggen. \cite{bezuijen_2000}

In Nederland kent men verschillende soorten schutsluizen met verschillende soorten sluishoofden, keringen en schutkolken. Deze sluizen van onderling hetzelfde type zijn ook in verschillende grootten te vinden, het is bijvoorbeeld ook mogelijk dat de aandrijving van de keringen anders is. \cite{bezuijen_2000}, \cite{arends_1994}

Om wat dieper in te gaan op de keringen van schutsluizen, deze zijn over het algemeen op te delen in twee categorieën: de \emph{translatie} keringen en de \emph{rotatie} keringen. \cite{bezuijen_2000}

De translatie keringen baseren hun bewegen op translatie, dit wil zeggen zijdelingse verplaatsing. Zo zijn er sluisdeuren te vinden die zich horizontaal bewegen, door het midden, (\emph{roldeur}) of bijvoorbeeld één deur die omhoog gaat (\emph{hefdeur}). Dit laatste type zorgt wel voor een maximale doorvaar hoogte, iets wat niet hoeft te zijn bij de horizontale translatie of de rotatie keringen. \cite{bezuijen_2000}, \cite{arends_1994}

Dan zijn er dus ook nog de rotatie keringen. Deze gaan uit, zoals de naam laat blijken, van een rotatie in het openen en sluiten van de kering. Rotatie keringen, ook wel \emph{puntdeuren} gaan uit van twee deuren. Deze twee deuren vormen een punt. Deze punt is dan gericht naar het hogere deel van het verloop. Dit zorgt er namelijk voor dat het hoger staande water tegen de punt in duwt, en verzekerd dat de deuren dicht zijn, en fysiek niet open kunnen gaan tijdens het op- of neerschutten. \cite{bezuijen_2000}, \cite{arends_1994} 

Om daadwerkelijk een boot op- of neer te kunnen schutten moet men wel eerst de mogelijkheid hebben om de schutkolk te vullen met water, en om dit water weer weg te kunnen laten lopen (ledigen). Het volbrengen van deze twee processen kan op een aantal manieren. Over het algemeen zullen sluizen met hefdeuren simpelweg hun hefdeuren langzaam op laten komen. Op deze manier kunnen deze sluizen geleidelijk de boten in de schutkolk op - en neerschutten. \cite{bezuijen_2000}

Er zijn ook sluizen te vinden met luiken in hun deuren. Deze luiken kunnen worden opengezet om de schutkolk of te vullen of te ledigen. Dit type van toe en afvoer is toe te passen op ieder type sluis. \cite{bezuijen_2000}

Als laatste heeft met ook nog de omloopriolen. Deze waterleidingen hebben meestal luiken op hun ingangen die opengezet kunnen worden om of water toe te voeren, of om water af te voeren. Door de druk die het water op zichzelf uitoefent is er over het algemeen geen pomp nodig om deze riolen te gebruiken. \cite{gww_2020}, \cite{bezuijen_2000}

In 1992 zijn type binnenvaart in klassen opgedeeld door het Conférence Européenne des Ministres des Transports, ook wel CEMT \cite{cemt_1992}. Nederland heeft wetgeving die er voor zorgt dat alle waterwegen in Nederland een CEMT classificatie hebben, zodat schippers weten wat voor route ze kunnen plannen met hun schip dat in een bepaalde klasse valt. Sluizen zijn hierbij geen uitzondering. In Nederland worden sluizen gebouwd naar een referentie schip, veelal is dit een schip gekozen door de persoon die de vaarweg beheert, en weet wat voor verkeer er over het algemeen door de vaarweg gaat. Tevens bepaalt dit schip ook de CEMT klasse van de vaarweg. \cite{cemt_1992}, \cite{rws_2020}

Sluizen zijn een kunstwerk dat je tegen kan komen tijdens het varen. Het gaan door een sluis brengt daarom ook verkeersregels met zich mee. Zo hebben sommige sluizen stoplichten, en zijn er borden te vinden aan de binnen- en buitenkant van de sluis. Aan weerszijden van de sluis zijn borden te vinden omtrent de regels in dergelijke sluis. De marifoon frequentie is er te vinden, samen met dingen als de verplichte snelheid, en of je wel of niet mag afmeren. \cite{zeilen_2013}

\end{document}
