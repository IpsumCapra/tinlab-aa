\documentclass[../verslag.tex]{subfiles}
\graphicspath{{\subfix{../images/}}}
\begin{document}

Rijkswaterstaat definieert een sluis als een scheiding tussen 2 wateren, met deuren, waarmee het mogelijk is om het waterpeil te beïnvloeden. Sluizen reguleren het waterpeil zodat schepen kunnen passeren. \cite{rws_2022}

In Nederland zijn veel verschillende soorten sluizen te vinden \cite{wsnoorderzijlvest_2021},\cite{arends_1994}. Zo zijn er doksluizen die worden gebruikt om het waterpeil in droogdokken en havens te regelen, en bijvoorbeeld spuisluizen die voorkomen dat de vaargeul waaraan ze zijn aangesloten dichtslibben door middel van een kom met zeewater die wordt gevuld tijdens vloed, en weer leeg wordt laten gelopen als het eb is \cite{wsnoorderzijlvest_2021},\cite{arends_1994}.

Dit onderzoek daarentegen focust zich op schutsluizen. Schutsluizen verschillen in veel andere sluizen in dat ze beschikken over twee of meer sluishoofden. Dit leidt ten gevolge ook tot dat schutsluizen een schutkolk hebben. De afgesloten ruimte die wordt gecreëerd als de keringen in de sluishoofden worden gesloten. De schutkolk kan gebruikt worden om geleidelijk het verval tussen de twee waterhoogten aan de buitenzijden van de sluishoofden te overbruggen. \cite{bezuijen_2000}

In Nederland kent men verschillende soorten schutsluizen met verschillende soorten sluishoofden, keringen en schutkolken. Deze sluizen van onderling hetzelfde type zijn ook in verschillende grootten te vinden, het is bijvoorbeeld ook mogelijk dat de aandrijving van de keringen anders is. \cite{bezuijen_2000}, \cite{arends_1994}

Translatie keringen en rotatie keringen

Water toe en afvoer

CEMT Klasse

Wetgeving (lichten, borden)



\end{document}
