\documentclass[../verslag.tex]{subfiles}
\graphicspath{{\subfix{../images/}}}
\begin{document}

Rijkswaterstaat definieert een sluis als een scheiding tussen 2 wateren, met deuren, waarmee het mogelijk is om het waterpeil te beïnvloeden. Sluizen reguleren het waterpeil zodat schepen kunnen passeren. \cite{rws_2022}

In Nederland zijn veel verschillende soorten sluizen te vinden \cite{wsnoorderzijlvest_2021},\cite{arends_1994}. Zo zijn er doksluizen die worden gebruikt om het waterpeil in droogdokken en havens te regelen, en bijvoorbeeld spuisluizen die voorkomen dat de vaargeul waaraan ze zijn aangesloten dichtslibben door middel van een kom met zeewater die wordt gevuld tijdens vloed, en weer leeg wordt laten gelopen als het eb is \cite{wsnoorderzijlvest_2021},\cite{arends_1994}.

Dit onderzoek daarentegen focust zich op schutsluizen. 

\end{document}
