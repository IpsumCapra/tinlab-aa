\documentclass[../literatuuronderzoek.tex]{subfiles}
\graphicspath{{\subfix{../images/}}}
\begin{document}


\subsubsection{Dagelijkse inspectie}
Als er kleine problemen zijn met de staat van een sluiscomplex, kan de veiligheid van de schippers in de sluis in gevaar komen. Daarom is het van belang om de sluizen \textbf{dagelijks} te inspecteren en eventuele afwijkingen, bijvoorbeeld vroegtijdige veroudering, te signaleren en de beschikbaarheid van de sluis in te schatten.
\cite{rws_belang_onderhoud}

De sluizen in het Twentekanaal mogen jaarlijks maximaal 20 uur buiten werking zijn ten gevolge van storingen. Dit is afgesproken door de minister van Infrastructuur en Waterstaat, en Rijkswaterstaat. Het doel is om de sluizen in goede conditie te houden en de sluizen zo min mogelijk uit gebruik te nemen.
\cite{rws_belang_onderhoud}


Bij een inspectie loopt de aannemer alle onderdelen langs.
\begin{itemize}
    \item Is de temperatuur goed?
    \item Gaan de deuren snel genoeg open?
    \item Is er roestvorming zichtbaar? \cite{wcm_onderhoud}
    \item De fysieke delen worden afgestoft en doorgesmeerd, daarnaast herstelt de aannemer kleine schades en doet hij storingenherstel.
    \item Omdat onze sluizen steeds vaker volledig geautomatiseerd zijn, wordt er wekelijks of maandelijks een analyse gemaakt van de programmatuur.
\end{itemize}
\cite{rws_belang_onderhoud}




\subsubsection{Door ontwikkelen onderhoud}
Sluizen worden langzamerhand steeds moderner. Ook het onderhoud wordt gemoderniseerd met bijvoorbeeld smart maintenance, waarbij sensoren in de apparatuur zijn geplaatst om de technische conditie bij te houden.
\cite{rws_belang_onderhoud}


\vspace{1em}

Bij Sluis Eefde hangen er op diverse onderdelen van de sluis meetapparatuur. Bijvoorbeeld werd er aan de hand van energiemetingen op de hoofdverdeling van de schuiven heel af en toe een afwijking gedetecteerd in de vermogens tussen de drie fases.
Toen een aannemer goed keek constateerde hij een onbalans in de motor. Deze onbalans wordt normaal gesproken pas opgemerkt als de tandwielkassen kapotgaan. Doordat dit nu tijdig werd ontdekt, werd de onbalans verholpen en is zo'n 20.000 euro aan schade en mogelijk ook het onderbreken van het gebruik voorkomen.
\cite{wcm_onderhoud}

Een volgende stap in dit proces zou kunnen zijn om ook monitoringsapparatuur voor het meten van trillingen te plaatsen, zoals nu al het geval is op de grote aandrijfmotoren en tandwielkasten. Op basis van energiedata en trillingsdata kunnen problemen nog sneller en accurater gevonden worden. Dit zou mogelijk maken dat het voorspellende model dan ook exact kan aanwijzen waar en op welk van de motoren het probleem zit en wat het is. Hierdoor hoeft een aannemer niet zelf meer fysiek te gaan zoeken.
\cite{wcm_onderhoud}




\paragraph{Weet wanneer corrosie begint}
Roest of corrosie treedt op bij objecten waar staal wordt gebruikt en die regelmatig in aanraking komen met water. Dit beïnvloedt de levensduur van een sluis. Bij de traditionele werkwijze worden objecten visueel geïnspecteerd op corrosie die men aan de bovenlaag waarneemt.
In feite ben je dan altijd te laat. Zodra corrosie zichtbaar is, is de ontwikkeling ervan moeilijk te stoppen zonder drastische maatregelen, zoals bijvoorbeeld 'zandstralen van de hele sluisdeur'.
Het herstellen van de onderliggende constructie en aanbrengen van een nieuwe beschermende laag enerzijds is al erg kostbaar. Daar bovenop komt dat de sluis voor een langere tijd onbruikbaar is.
\cite{wcm_onderhoud}

C-Cube, specialist in corrosie, heeft het doe on de kennis en kunde over corrosie te vertalen naar een onlinedienst; de 'predictive corrosion service'. Hiermee wordt berekend wanneer corrosie gaat optreden, jaren vooruit. Hiermee kunnen we onnodige schade door corrosie voor zijn en zorgt voor een kostenbesparing rond de 40\%.
De eerste toepassing van deze dienst is inmiddels succesvol getest binnen het project Sluis Eefde.
\cite{wcm_onderhoud}


\end{document}