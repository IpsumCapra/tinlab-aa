\documentclass[../verslag.tex]{subfiles}
\graphicspath{{\subfix{../images/}}}
\begin{document}


Het model is gerepresenteerd als een Kripke structuur. Hiervoor is UPPAAL gebruikt. UPPAAL is een programma dat is gemaakt om modellen te maken en te valideren.

Het meest overzichtelijk is om het model van de sluis op te delen in losse onderdelen. Deze onderdelen kunnen dan gecombineerd worden tot een complete sluis.


\subsection{Onderdelen} 

\subsubsection{Sluisdeur}
In een sluis zijn altijd twee sluisdeuren aanwezig.

Een sluisdeur heeft 3 states: Dicht, Bewegende en Open.


\subsubsection{Stoplicht}
In een sluis zijn altijd twee stoplichten aanwezig, een stoplicht per sluisdeur.

Een stoplicht heeft 2 states: Rood en Groen.


\subsubsection{Schutkolk}
Een schutkolk heeft 3 states: LinkerNiveau, Schuttende en RechterNiveau.

Als het water in de schutkolk op het linker of het rechter niveau is kunnen boten uit en in de schutkolk varen. Het uit- en invaren van boten kan gedaan worden met transities op de states.

Als de schutkolk schuttende is, kunnen er geen boten in- of uitvaren.


\subsubsection{Sluiscontroller}

De sluiscontroller heeft 6 states: GeschutL, DeurLOpen, DeurLDicht, GeschutR, DeurROpen en DeurRDicht.

Deze states worden ook op deze volgorde geactiveerd om een cyclus van de sluis uit te voeren. Deze cyclus blijft zich herhalen.



\subsection{Totaal}

Het model bestaat uit vier verschillende onderdelen: sluisdeur, stoplicht, schutkolk en sluiscontroller. Deze werken samen door transities te synchroniseren. Het stoplicht gaat bijvoorbeeld op groen als de bijbehorende sluisdeur opengaat.

Timing

Het model voor de sluis is modulair. Het aantal boten dat in de sluis wordt toegelaten kan aangepast worden en het is ook mogelijk om meerdere sluizen te maken.


\end{document}
